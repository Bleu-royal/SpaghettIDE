\documentclass[a4paper,12pt]{article} %style de document
\usepackage[utf8]{inputenc} %encodage des caractères
\usepackage[french]{babel} %paquet de langue français
\usepackage[T1]{fontenc} %encodage de la police
\usepackage[top=2cm,bottom=2cm,left=2cm,right=2cm]{geometry} %marges
\usepackage{graphicx} %affichage des images
\usepackage{wrapfig} %Images à côté du texte
\usepackage{url}

\begin{document} %début du document

%page de garde
\begin{titlepage}
	\begin{center}
		\Large{Année universitaire 2016-2017}\\
		\Large{Université de Caen Basse-Normandie}\\[1cm]
		
		\huge{Cahier des charges du projet de TPA :}\\
		L'IDE\\
		\vspace{3cm}
		Alexis Carreau\\
		Thomas Lécluse\\
		Emma Mauger\\
		Théo Sarrazin\\
	\normalsize{\textit{ ~ L2 Informatique}}\\
		\medskip
		\vspace{2cm}
		
	\end{center}
\end{titlepage}

\tableofcontents
\newpage

\section{Définition de l'IDE}
	L'IDE -\normalsize{\textit{Integrated Developpement Environement}}- est un logiciel qui fournit des facilités au programmeur pour le développement de logiciels, d'applications ou encore de sites internets. Il va ainsi maximiser la productivité du programmeur.\\ 	
	Pour se faire, l'IDE utilise l'autocomplétion. C'est à dire qu'il reconnaît toutes sortes de balises, de méthodes, de mots-clés et va compléter automatiquement ce qu'écrit l'utilisateur s'il le reconnaît. Il y a aussi un accès facile à une documentation complète, et un aperçu en direct du programme en train d'être écrit.
	
\section{Que contient-il ?}
	Un IDE contient entre autre :
	\begin{itemize}
		\item Un éditeur de texte, pour écrire et modifier des fichiers dans un langage traité par l'IDE.
		\item Un debugger, qui va aider à analyser et repérer des bugs dans un programme en l'exécutant pas-à-pas. On peut observer donc étape par étape les valeurs prises par des variables afin de voir la source du bug.
		\item Un interpréteur, qui analyse, traduit (si la syntaxe est correcte) et exécute à partir du fichier source le code.
	\end{itemize}
	
\section{Langages supportés}
	Un IDE peut supporter tous types de langages. Pour notre part, nous allons commencer avec du C.
	
\section{Notre IDE}
	On doit pouvoir :
	\begin{itemize}
		\item Créer un projet, le modifier, le supprimer.
		\item Créer un répertoire.
		\item Créer un document dans un projet et l'enregistrer, le supprimer et l'éditer.
		\item Ouvrir un projet / un document.
		\item Gérer l'autocomplétion pour les balises C.
		\item Repérer ou mettre en valeur des balises.
		\item Replier ou masquer des paragraphes en C et la liste des propriétés.
		\item Afficher un aperçu en direct du site en réalisation.
	\end{itemize}


\end{document}