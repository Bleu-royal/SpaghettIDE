\documentclass[a4paper,12pt]{article}
\usepackage[utf8]{inputenc}
\usepackage[french]{babel}
\usepackage[T1]{fontenc}
\usepackage[top=2cm,bottom=2cm,left=2cm,right=2cm]{geometry}
\usepackage{graphicx}
\usepackage{wrapfig}
\usepackage{url}

\begin{document}

\begin{titlepage}
	\begin{center}
		\Large{Année universitaire 2016-2017}\\
		\Large{Université de Caen Basse-Normandie}\\[1cm]
		
		\huge{Documentation PySide:}\\
		Comment utiliser PySide pour notre IDE ?\\
		\vspace{3cm}
		Alexis Carreau\\
		Thomas Lécluse\\
		Emma Mauger\\
		Théo Sarrazin\\
	\normalsize{\textit{ ~ L2 Informatique}}\\
		\medskip
		\vspace{3cm}

	\begin{figure}[!h]

			\begin{center}

				\includegraphics[scale=1.5]{"images/pysidelogo"}

			\end{center}

		\end{figure}
		
	\end{center}
\end{titlepage}


\tableofcontents

\newpage

\section{Introduction}

Nous allons utiliser PySide pour faire notre interface graphique. PySide est une librairie qui fait l'intermédiaire entre Python et QT. QT est une librairie d'inferface graphique destinée au C. Nous avons fait ce choix car deux d'entre nous connaissaient déjà QT. De plus, il y a une documentation bien expliquée, explicite, et complète (cf https://wiki.qt.io/PySide). 

N'ayant pas fini notre projet, nous allons compléter cette documentation au fur et à mesure de l'avancement.

\section{Fonctionnement}

	\subsection{En théorie}

		Pour réaliser une interface graphique, on ré-implémente des classes de bases de QT, pour les adapter à nos besoins. Les classes créées héritent donc des superclasses et bénéficient de leurs attributs et de leurs méthodes. Cela nous permet de créer des widgets, qui sont des élements constituant l'interface et de les modifier à convenance.

	\subsection{En pratique} 

		Quelques exemples :

		Tout d'abord, il faut importer les modules nécessaires au bon fontionnemment de l'interface graphique :

		\begin{figure}[!h]

			\begin{center}

				\includegraphics[scale=1]{"images/Import"}

				\caption{Importation des modules}

			\end{center}

		\end{figure}

		Puis, on créé la classe Fenêtre qui hérite de la classe QWidget et on appelle le constructeur du parent.

		\begin{figure}[!h]

			\begin{center}

				\includegraphics[scale=0.8]{"images/QWidget"}

				\caption{Définition de la structure de la classe Fenêtre}
			\end{center}

		\end{figure}
		\newpage
		Ensuite, on fait appel à la classe QApplication pour instancier une nouvelle application. Puis, on affiche l'interface graphique avec notre classe Fenêtre qui hérite de la classe QWidget que nous avons créé. 

		\begin{figure}[!h]

			\begin{center}

				\includegraphics[scale=0.8]{"images/QApplication"}

				\caption{Création de l'interface}

			\end{center}

		\end{figure}

		Un exemple d'utilisation de widget et le QTextEdit qui est une zone de texte éditable sur plusieurs lignes. On peut le personaliser en ajoutant par exmeple une police, la taille et la couleur du texte.

		\begin{figure}[!h]

			\begin{center}

				\includegraphics[scale=0.8]{"images/QTextEdit"}

				\caption{Création et personnalisation du QTextEdit}
			\end{center}

		\end{figure}

		Une fois les objets créés, il faut les positionner. Pour cela, on utilise les Layout qui permet de positionner les différentes parties constituant notre interface graphique. Dans notre cas le self.layout est une instance de la classe QGridLayout. Enfin self.show permet d'afficher la fenêtre. 

		\begin{figure}[!h]

			\begin{center}

				\includegraphics[scale=0.8]{"images/Layout"}

				\caption{Positionnement et affichage des widgets}
			\end{center}

		\end{figure}

		



\end{document}